\section{Research Field and Motivation}\label{sec:research_mot}
\subsection{Motivation}\label{subsec:motivation}
In an era where manual assembly is no longer possible in the majority of developed countries, it is necessary for companies offering alternative solutions to stand out from their competitors. Manufacturers of assembly lines must modify the capabilities of their robots in order to be more competitive than the relocation of the assembly to countries with cheaper labour. In order to stand out, companies such as Mikron SA, seek to innovate grippers. The goal of this project would be to conceive and develop a novel and innovative type of Smart gripper. This small part plays a primordial role in the dynamics of the robot. Being at the end of the arm, a small gain in weight of this part would have great consequences on the acceleration and the maximum speed that the robot will be able to reach.

The \emph{Laboratory of Integrated Actuators}, along with a Swiss company, \emph{Mikron SA}, intend to develop a novel type of gripper that will harness the high work output per volume of smart materials. One such promising category of smart materials are the Shape Memory Alloys (SMA). This gripper would be designed to be lightweight and compact so as to be used as a pick and place gripper in clean room application. The project strives to investigate an innovative technology that will exploit the characteristics of this smart material to create an actuation system that is highly responsive, dynamic, lightweight and compact. These objective will motivate a doctoral thesis that is challenging and innovative.

\subsection{Research Field}\label{subsec:research_field}
Currently, few industrial actuators based on SMAs have been seen. This project first has to demonstrate the potential of such actuators as a gripper. Then optimisation of the right topology should reveal the full potential of SMA actuators in terms of time response and stroke. Heating the element is not the greatest challenge, the cooling time of the material must also be taken into account. It is important to note that it is often harder to dissipate heat rather than accumulate it. Thus, various heating and cooling solutions will be exploited and investigated during this project. Another challenge to overcome would be the limited fatigue life of such smart materials. The gripper should at the very least have a fatigue life in the order of $10^6$ cycles. The last challenge to over overcome would be to meet the requirements set by the currently used pneumatic grippers in industry.

This investigation will consider the thermal and mechanical aspect of the material so as to create a highly responsive and dynamic actuator capable of achieving the required force output and stroke of traditional grippers. The conception of the gripper will include optimization of the geometrical topology of the SMA blades and the use of bistable systems such as buckled beams to create dynamic and high stroke actuators. The command strategies of the actuator will also be an important area of study due to the fact that the actuation of the SMA will involve the precise control of its temperature and resistance.

The symbiosis of the bistable system and the SMA technology can result in an innovative gripper system. The interplay between these two domains will be studied and explored.

\subsection{Specifications of the gripper}\label{subsec:specifications}
The specifications of the actuator were obtained by comparing it with the specifications of the Schunk MPG-25 which is a common pneumatic gripper used in industry, more specifically the primary gripper used by Mikron SA.

\begin{table}[H]%{r}{5.5\textwidth}
  \centering
  \footnotesize
  \caption{Specifications of the required actuator}
  \label{tab:specs}
  \begin{tabu} to 0.5\textwidth {>{\bfseries}X[l, 3]  >{\bfseries}X[l, 1] X[r,1]}
      \tableHeaderStyle{tableRed}
      Criteria & Units & Value\\
      Stroke & mm & 3\\
      Grip force & N & 5\\
      Commutation time & ms & 50 - 100\\
      Weight & g & 100\\
      Precision/Repeatability & mm & 0.02\\
      Number of stable positions & \# & 2\\
      Volume & mm$^3$ & 120 \\
  \end{tabu}
\end{table}

Using the above specifications, the energy that the smart material must supply can be calculated to be approximately 15 mJ. This implies, for example, that for materials such as SMAs like NiTiNOL, the application would require only a volume of 1.5 mm$^3$.
