\section{Research Field and Motivation}\label{sec:research_mot}
\subsection{Motivation}\label{subsec:motivation}
%
% Currently, the Mikron workshops are equipped with pneumatic grippers, the Schunk MPG-25 as shown in figure 1. This type of gripper is currently the best of the market with an excellent power to weight ratio. But the force provided by the gripper is too high by a factor of 10 compared to the needs of the company. With the cheapest version of the MPG-25 series, it is necessary to pass two nozzles of the air injection monitor, next to the moving part of the robot, to the gripper, thus complexifying the design of the arms compared to a power supply. Furthermore, any pneumatic installation requires an air pump. In general, the capacity of the latter depends on the number of grippers used at the same time. When the number of active grippers drops, the pump always runs at the same speed, resulting in a loss of efficiency. The last critical point of this technology is its adaptation in a clean environment. Since these grippers release a certain amount of air during each change of position. It is imperative to install a high-quality air filter that will have to be changed regularly to ensure that the installation complies with the cleanliness standards in some areas such as biomedical.
%
% On the market, there are a wide variety of gripper that use different gripping techniques. Some operate using air suction, which creates a pressure difference under a tube and acts like a suction cup to grip objects. They are particularly useful for flat and smooth objects. Due to the low flexibility with regards to the type of parts that can be handled, these types of grippers are not very widespread. In the same manner, there are electromagnetic grippers where magnetic parts are manipulated. These grippers can be sized specifically to serve different sizes of pieces.
%
% Shape memory alloys (SMA) are alloys that have the particular property of not respecting the Hooke’s law with respect to deformation. The material possesses two distinctive phases, the Austenite and Martensite phases. The temperatures at which the material transforms into the two phases are known as the Austenite and Martensite temperature. By heating the material above this temperature, it is possible, after plastic deformation, to bring it back to its initial position. Furthermore, this position can be precisely chosen.
%
% In order to have a system allowing Mikron SA to maximise the rate of their assembly, we focused on optimizing the weight of the proposed actuator. By comparing the response times obtained by some laboratories, including MIT with electrically heated SMAs, we decided to actuate using thin SMA blades. We plan to differentiate ourselves from traditional SMA actuators by developing a hybrid heating solution that combines internal Joule effect heating and an external heater. We propose that this would significantly reduce the transition time of the SMA and optimize the time response. We also decided to use a mobile magnet to increase the efficiency and performance of the actuator as illustrated in Figure 6 in the annex. On either side of the actuator, we intend to place ferromagnetic pieces which will attract the magnet with the certain amount of force. This setup will give us the two stable positions required. A mechanism with several NiTiNOL blades will deform when the magnet arrives and these blades will be used to alternate between the two stable positions by applying the initial force to actuate the magnet. By heating the blades, the SMA will change phase and their stiffness will increase causing it to apply a force onto the magnet. Once this force exceeds the attractive force of the magnet, the mobile magnet will be attracted and be displaced to the other stable position on the opposite side. There are have been multiple studies into SMAs, such as [1], that show that SMAs have a very high power to weight ratio when compared to other types of actuation. The proposed mechanism allows us to match the product requirements while at the same time being extremely lightweight.
%
%  Moreover, as suggesting in [3], in order to have a sufficient lifecycle, the maximum deformation of the material should not exceed 3\%. We have decided to dedicate a work package for this critical area of the project. There are have been some research that looks into different methods by which SMA blades can be heated and cooled such as [4]. Using printed coils and passing an electric current, we can use the Joule effect to effectively heat the blade. By extensively studying different heating and cooling solutions, we can hope to obtain an optimized solution that would allow us to mitigate this risk.
In an era where manual assembly is no longer possible in the majority of developed countries, it is necessary for companies offering alternative solutions to stand out from their competitors. Manufacturers of assembly lines must modify the capabilities of their robots in order to be as competitive as possible compared to the relocation of the assembly in countries with cheaper labour. In order to stand out, companies such as Mikron SA, seek to increase the efficiency of their installations while preserving the precision appreciated by their customers. The goal of this project would be to conceive and develop a novel and innovative type of Smart gripper. This small part plays a primordial role in the dynamics of the robot. Being at the end of the arm, a small gain in weight of this part would have great consequences on the acceleration and the maximum speed that the robot will be able to reach.

The \emph{Laboratory of Integrated Actuators}, along with a Swiss company, \emph{Mikron SA}, intend to develop a novel type of gripper that will harness the high work output per volume of Shape Memory Alloys (SMA). This gripper would be designed to be lightweight and compact so as to be used as a pick and place gripper in clean room application. The project strives to create an innovative technology that will exploit the characteristics of this smart material to create an actuator that is highly responsive, dynamic, lightweight and compact. These objective will motivate a doctoral thesis that is challenging and innovative.

\subsection{Research Field}\label{subsec:research_field}
This project will focus on high response SMA actuation with a large stroke. Currently, few industrial actuators based on SMAs have been seen. This project first has to demonstrate the potential of such actuators as a gripper. Then optimisation of the right topology should reveal the full potential of SMA actuators in terms of time response and stroke. Heating the element is not the greatest challenge, the cooling time of the material must also be taken into account. It is important to note that it is often harder to dissipate heat rather than accumulate it. Thus, various heating and cooling solutions will be exploited and investigated during this project. Another challenge to overcome would be the limited fatigue life of such smart materials. The gripper should at the very least have a fatigue life in the order of $10^6$ cycles. The last challenge to over overcome would be to meet the requirements set by the currently used pneumatic grippers in industry.

This investigation will consider the thermal and mechanical aspect of the material so as to create a highly responsive and dynamic actuator capable of achieving the required force output of traditional grippers. The main focus of the research will be the thermal and mechanical optimization. The conception of the gripper will include optimization of the geometrical topology of the SMA blades and the use of bistable systems such as buckled beams to create dynamic and high stroke actuators. The command strategies of the actuator will also be an important area of study due to the fact that the actuation of the SMA will involve the precise control of its temperature and resistance.

The symbiosis of the bistable system and the SMA technology can result in an innovative gripper system. The interplay between these two domains will be studied and explored in this thesis.

\subsection{Specifications}\label{subsec:specifications}
The specifications of the actuator were obtained by comparing it with the specifications of the Schunk MPG-25 which is the most common pneumatic gripper used in industry, more specifically the primary gripper used by Mikron SA.
\begin{table}[h]
  \centering
  \caption{Specifications of the required actuator}
  \label{tab:specs}
  \begin{tabu} to 0.7\textwidth {>{\bfseries}X[l, 3]  >{\bfseries}X[l, 1] X[r,1]}
      \tableHeaderStyle{tableRed}
      Criteria & Units & Value\\
      Stroke & mm & 3\\
      Grip force & N & 5\\
      Commutation time & ms & 50 - 100\\
      Weight & g & 100\\
      Precision/Repeatability & mm & 0.02\\
      Number of stable positions & \# & 2\\
      Volume & mm$^3$ & 120 \\
  \end{tabu}
\end{table}

Using the above specifications, the energy that the smart material must supply can be calculated to be approximately 15 mJ. This implies, for example, that for materials such as SMA, the application would require only a volume of 1.5 mm$^3$.
% The \emph{Laboratoire d'actionneurs int\'{e}gr\'{e}s} (LAI), along an international NGO and two Swiss companies, are developing a novel suit concept that will be paired with a high-reliability ventilation system. This protective suit is designed to \emph{refresh} and \emph{protect} professionals of the health sector, who battle infectious diseases in tropical climates and in remote areas of developing countries. This context induces high constraints in terms of harsh working environment, limited maintenance possibilities, and financial resources.
%
% The cooperation strives to develop a suit that is portable, robust and economically accessible. Its ventilation system and its corresponding drive, complying with the aforementioned characteristics, will motivate a doctoral thesis at the \emph{EPFL}. To this end, a miniaturized electric drive with magnetics bearings seems suitable for the constraining operating conditions, given its close-to-none maintenance needs, construction simplicity and thus their miniaturization potential.
% \subsection{Research Field}\label{subsec:research_field}
% This investigation will focus upon high-speed rotatory electric drives and magnetic bearings. It will also exhibit an industrial profile, since it is aspires to set the cornerstones of new product trends of its supporting companies.
%
% Therefore, this projects considers the conception, construction, and commissioning of a \emph{magnetically levitated }(mag-lev) motor. The correct deliberation of practical and theoretical aspects will determine the success of this research.
%
% For the conception, one of the supporting companies will define what type of ventilator topology is needed for the proper refreshing inside suit.  The selection of motor topology has to match ventilator geometry and its needed performance. The design of mag-lev drives diverges from classical motor design, since not only drive but \emph{bearing performance} has to be considered. On top, drive and bearing performance usually trade-off with one another, so an optimum has to be found.
%
% This conception has to consider the feasibility of actual manufacturing procedures.  The assembly of the drive will be challenging because the mechanical tolerances of manufacturing methods may be relatively large for the desired \emph{millimetric-scale} of the prototype. Homogenously-cut mechanical pieces, thin impeller elements and evenly-spaced sensor placement are crucial for the match between theory and the practice.
%
% Commissioning is the culmination of the conception and construction. It will involve read-in of various electric signals, its processing, and the generation of actuator commands, all at a digital level. Given that the motor will rotate at high-speeds, the code must be time-optimized. Various closed-loop plants will have to be tuned, for speed, rotor position and current control, and non-idealities will also have to be compensated for.
% \subsection{Specifications}\label{subsec:specs}
% To generate a scientific breakthrough, and simultaneously comply with the industrial requirements, the motor needs to meet the following requirements.
% \begin{center}
% \begin{tabular}{c|c|c|c}
%   % after \\: \hline or \cline{col1-col2} \cline{col3-col4} ...
%   Mass & Power & Rotational Speed & Axial length \\ \hline
%   $< 150$ g & $<30$ W & $> 100$ krpm & $< 50$ mm \\
% \end{tabular}
% \end{center}
