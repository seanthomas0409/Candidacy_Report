\section{State of the Art}\label{sec:sota}
This section will discuss the existing smart materials that are currently studied and used in the domain of actuators. The different techniques and integration systems will also be presented with the goal to compare the various approaches currently used.

\subsection{Smart materials}
In the field of engineering, ranging from haptics, automation and bio-medical fields, there has been a need  to create actuators that are lightweight, compact and having force output. This creates a need for materials that can deliver high forces and strokes while remaining light and small meaning that the materials need to have a high work output.

On the basis of creating an actuator that can meet the demands of the currently implemented strategies while at the same time pushing the limits of the current technology, a thorough investigation of the available smart materials must be conducted. These materials have the ability to react to an external stimulus such as thermal electrical or magnetic and are thus referred to as \emph{smart} or \emph{active materials}. These materials have an inherent property that allows them to be exploited with a specific external stimulus so as to alter their mechanical characteristics or to create self-sensing technology.

There exist numerous types of smart materials and based on their properties, they can be classified into many types such as \cite{damodharan_review_2018}:
\begin{itemize}
	\item Piezoelectric materials
	\item Magneto-strictive materials
	\item Electro-active polymers
	\item Shape Memory Alloys
\end{itemize}

The aim of this project is to adapt these aforementioned smart materials to harness their specific behaviour and fabricate smart actuators. This implies that the system that incorporates the material is equally critical for the conception of the actuator. This section of the report will delved into different strategies used to harness the specific behaviours of the smart materials.

\subsubsection{Piezoelectric Materials}
Piezoelectric materials are a subgroup of smart materials that have the capability to produce voltages when a stress is applied. This behaviour is can also be expressed in the opposite direction i.e. a strain can be generated using an electric field.
